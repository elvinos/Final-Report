\section{Welcome to Marxico}\label{welcome-to-marxico}

\textbf{Marxico} is a delicate Markdown editor for Evernote. With
reliable storage and sync powered by Evernote, \textbf{Marxico} offers
greate writing experience.

\begin{itemize}
\tightlist
\item
  \textbf{Versatile} - supporting code highlight, \emph{LaTeX} \& flow
  charts, inserting images \& attachments by all means.
\item
  \textbf{Exquisite} - neat but powerful editor, featuring offline docs,
  live preview, and offering the {[}desktop client{]}{[}1{]} and offline
  {[}Chrome App{]}{[}2{]}.
\item
  \textbf{Sophisticated} - deeply integrated with Evernote, supporting
  notebook \& tags, two-way bind editing.
\end{itemize}

\begin{center}\rule{0.5\linewidth}{\linethickness}\end{center}

\subsection{Introducing Markdown}\label{introducing-markdown}

\begin{quote}
Markdown is a plain text formatting syntax designed to be converted to
HTML. Markdown is popularly used as format for readme files, \ldots{} or
in text editors for the quick creation of rich text documents. -
\href{http://en.wikipedia.org/wiki/Markdown}{Wikipedia}
\end{quote}

As showed in this manual, it uses hash(\#) to identify headings,
emphasizes some text to be \textbf{bold} or \emph{italic}. You can
insert a \href{http://www.example.com}{link} , or a
footnote{[}\^{}demo{]}. Serveral advanced syntax are listed below,
please press \texttt{Cmd\ +\ /} to view Markdown cheatsheet.

\subsubsection{LaTeX expression}\label{latex-expression}

\[ x = \dfrac{-b \pm \sqrt{b^2 - 4ac}}{2a} \]

\begin{quote}
\textbf{Note:} Currently it is only partially supported. You can't
toggle checkboxes in Evernote. You can only modify the Markdown in
Marxico to do that. Next version will fix this.
\end{quote}

\subsubsection{Dancing with Evernote}\label{dancing-with-evernote}

\paragraph{Notebook \& Tags}\label{notebook-tags}

\textbf{Marxico} add syntax to select notebook and set tags for the
note. After typing \texttt{@(}, the notebook list would appear, please
select one from it.

\paragraph{Title}\label{title}

\textbf{Marxico} would adopt the first heading encountered as the note
title. For example, in this manual the first line
\texttt{Welcome\ to\ Marxico} is the title.

\paragraph{Quick Editing}\label{quick-editing}

Note saved by \textbf{Marxico} in Evernote would have a red ribbon
button on the top-right corner. Click it and it would bring you back to
\textbf{Marxico} to edit the note.

\begin{quote}
\textbf{Note:} Currently \textbf{Marxico} is unable to detect and merge
any modifications in Evernote by user. Please go back to
\textbf{Marxico} to edit.
\end{quote}

\paragraph{Data Synchronization}\label{data-synchronization}

While saving rich HTML content in Evernote, \textbf{Marxico} puts the
Markdown text in a hidden area of the note, which makes it possible to
get the original text in \textbf{Marxico} and edit it again. This is a
really brilliant design because:

\begin{itemize}
\tightlist
\item
  it is beyond just one-way exporting HTML which other services do;
\item
  and it avoids privacy and security problems caused by storing content
  in a intermediate server.
\end{itemize}

\begin{quote}
\textbf{Privacy Statement: All of your notes data are saved in Evernote.
Marxico doesn't save any of them.}
\end{quote}

\paragraph{Offline Storage}\label{offline-storage}

\textbf{Marxico} stores your unsynchronized content locally in browser
storage, so no worries about network and broswer crash. It also keeps
the recent file list you've edited in
\texttt{Document\ Management(Cmd\ +\ O)}.

\begin{quote}
\textbf{Note:} Opthough browser storage is reliable in the most time,
Evernote is born to do that. So please sync the document regularly while
writing.
\end{quote}

\subsection{Shortcuts}\label{shortcuts}

Help \texttt{Cmd\ +\ /} Sync Doc \texttt{Cmd\ +\ S} Create Doc
\texttt{Cmd\ +\ Opt\ +\ N} Maximize Editor \texttt{Cmd\ +\ Enter}
Preview Doc \texttt{Cmd\ +\ Opt\ +\ Enter} Doc Management
\texttt{Cmd\ +\ O} Menu \texttt{Cmd\ +\ M}

Bold \texttt{Cmd\ +\ B} Insert Image \texttt{Cmd\ +\ G} Insert Link
\texttt{Cmd\ +\ L} Convert Heading \texttt{Cmd\ +\ H}

\subsection{About Pro}\label{about-pro}

\textbf{Marixo} offers a free trial of 10 days. After that, you need to
\href{http://marxi.co/purchase.html}{purchase} the Pro service.
Otherwise, you would not be able to sync new notes. Previous notes can
be edited and synced all the time.

\subsection{Credits}\label{credits}

\textbf{Marxico} was first built upon {[}Dillinger{]}{[}5{]}, and the
newest version is almost based on the awesome {[}StackEdit{]}{[}6{]}.
Acknowledgments to them and other incredible open source projects!

\subsection{Feedback \& Bug Report}\label{feedback-bug-report}

\begin{itemize}
\tightlist
\item
  Twitter: {[}@gock2{]}{[}7{]}
\item
  Email:
  \href{mailto:hustgock@gmail.com}{\nolinkurl{hustgock@gmail.com}}
\end{itemize}

\begin{center}\rule{0.5\linewidth}{\linethickness}\end{center}

Thank you for reading this manual. Now please press \texttt{Cmd\ +\ M}
and click \texttt{Link\ with\ Evernote}. Enjoy your \textbf{Marxico}
journey!
